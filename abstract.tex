This dissertation details the development of an open source, highly frequency domain multiplexed (FDM) readout for large-format arrays of superconducting lumped-element kinetic inductance detectors (LEKIDs). The system architecture is designed to meet the requirements of current and next generation balloon-borne and ground-based submillimeter (sub-mm), far-infrared (FIR) and millimeter-wave (mm-wave) astronomical cameras, whose science goals will soon drive the pixel counts of sub-mm detector arrays from the kilopixel to the megapixel regime. The in-flight performance of the readout system was verified during the summer, 2018, flight of ASI's OLIMPO balloon-borne telescope, from Svalbard, Norway. This was the first flight for both LEKID detectors and their associated readout electronics. In winter 2019/2020, the system will fly on NASA's long-duration Balloon Borne Large Aperture Submillimeter Telescope (BLAST-TNG), a sub-mm polarimeter which will map the polarized thermal emission from cosmic dust at 250, 350 and 500 microns (spatial resolution of 30$\arcsec$, 41$\arcsec$, and 59$\arcsec$). It is also a core system in several upcoming ground based mm-wave instruments which will soon observe at the 50~m Large Millimeter Telescope (e.g., TolTEC, SuperSpec, MUSCAT), at Sierra Negra, Mexico.

The design and verification of the FPGA firmware, software and electronics which make up the system are described in detail. Primary system requirements are derived from the science objectives of BLAST-TNG, and special considerations (e.g., SWAP-C) for balloon-borne platforms are taken into account. The system was used to characterize the instrumental performance of the BLAST-TNG receiver and detector arrays in the lead-up to the 2019/2020 flight attempt from McMurdo Station, Antarctica. The results of this characterization are interpreted by applying a parametric software model of a LEKID detector to the measured data in order to estimate important system parameters, including the optical efficiency, optical passbands and sensitivity.

The role that magnetic fields (B-fields) play in shaping structures on various scales in the interstellar medium is one of the central areas of research which is carried out by sub-mm/FIR observatories. The Davis-Chandrasekhar-Fermi Method (DCFM) is applied to a BLASTPol 2012 map (smoothed to 5$\arcmin$) of the inner $\sim$1.25~deg$^{2}$ of the Carina Nebula Complex (CNC, NGC 3372) in order to estimate the strength of the B-field in the plane-of-the-sky ($B_{\mathrm{POS}}$). The resulting map contains estimates of $B_{\mathrm{POS}}$ along several thousand sightlines through the CNC. This data analysis pipeline will be used to process maps of the CNC and other science targets which will be produced during the upcoming BLAST-TNG flight. A target selection survey of five nearby external galaxies which will be mapped during the flight is also presented.

%\end{abstract}
